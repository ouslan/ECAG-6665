\documentclass[12pt]{beamer}

% ****************
% ***** INFO *****
% ****************
\usepackage[english]{babel}
\title[]{Assignment 1}
\subtitle{ECAG-6665: Applied Econometrics}
\author[Name Surname]{Alejandro Ouslan}
\institute[UPR.png]{University of Puerto Rico}
\date{} % or \today

% *******************
% ***** PROJECT *****
% *******************
% main color: to black
\definecolor{main}{HTML}{000000}
\setbeamercolor{structure}{fg=main}

% *****************
% ***** THEME *****
% *****************
\usepackage{helvet}
\renewcommand{\familydefault}{\sfdefault}
\setbeamertemplate{frametitle continuation}{\gdef\beamer@frametitle{}}
\setbeamertemplate{footline}{}
\setbeamertemplate{navigation symbols}{}
\usepackage{csquotes}
\usepackage[backend=biber,style=numeric]{biblatex}
\addbibresource{sample.bib} % Link to the bibliography file

% *****************
% ***** CODE *****
% *****************
\usepackage{listings}
\lstdefinestyle{java}{
	backgroundcolor=\color{white},
	basicstyle=\ttfamily\scriptsize,
	breaklines=true,
	commentstyle=\color{gray},
	keywordstyle=\color{blue},
	stringstyle=\color{magenta},
	identifierstyle=\color{black},
	numberstyle=\color{gray},
	language=Java
}
\lstdefinestyle{cpp}{
	backgroundcolor=\color{white},
	basicstyle=\ttfamily\scriptsize,
	breaklines=true,
	commentstyle=\color{gray},
	keywordstyle=\color{blue},
	stringstyle=\color{magenta},
	identifierstyle=\color{black},
	numberstyle=\color{gray},
	language=C++
}
\lstdefinestyle{py}{
	backgroundcolor=\color{white},
	basicstyle=\ttfamily\scriptsize,
	breaklines=true,
	commentstyle=\color{gray},
	keywordstyle=\color{blue},
	stringstyle=\color{magenta},
	language=Python
}
\lstdefinestyle{js}{
	backgroundcolor=\color{white},
	basicstyle=\ttfamily\scriptsize,
	breaklines=true,
	commentstyle=\color{gray},
	keywordstyle=\color{blue},
	stringstyle=\color{magenta},
	identifierstyle=\color{black},
	numberstyle=\color{gray},
	language=JavaScript,
	escapechar=@
}
\lstdefinestyle{sh}{
	basicstyle=\ttfamily\scriptsize,
	breaklines=true,
	commentstyle=\color{gray},
	keywordstyle=\color{blue},
	stringstyle=\color{magenta},
	identifierstyle=\color{black},
	numberstyle=\color{gray},
	language=bash
}

% **********************
% ***** ALGORITHMS *****
% **********************
\usepackage{algorithm}
\usepackage{algpseudocode}

% *****************
% ***** UTILS *****
% *****************
\usepackage{xcolor}

% ********************
% ***** DOCUMENT *****
% ********************
\begin{document}

% **********************
% ***** TITLEPAGE ******
% **********************
\begin{frame}{}
	\vspace{\fill}

	\includegraphics[width=0.16\linewidth]{uprm_logo.png}

	\vspace{\fill}

	\Large
	\color{main}
	\inserttitle

	\medskip

	\large
	\color{black}
	\insertsubtitle

	\vspace{\fill}

	\footnotesize
	\insertinstitute

	\vspace{\fill}

	\textbf{Author:} \insertauthor

	\medskip

	\insertdate

	\vspace{\fill}
\end{frame}

% *****************
% ***** START *****
% *****************
\begin{frame}[allowframebreaks]{Typography}
	\begin{itemize}
		\item \textbf{Bold}
		\item \textit{Italic}
		\item \texttt{Monospaced}
		\item \underline{Underlined}
		\item \href{https://example.com/}{\underline{\color{main}{Link}}}
	\end{itemize}
\end{frame}

\begin{frame}[allowframebreaks]{Lists}
	Itemize:

	\begin{itemize}
		\item \textbf{Item 1:} Example of an item in an itemize list.
		\item \textbf{Item 2:} Another item demonstrating the use of itemize.
		\item \textbf{Item 3:} Yet another example to show list formatting.
	\end{itemize}
\end{frame}

\begin{frame}[allowframebreaks]{Research Question}

\end{frame}

\begin{frame}[allowframebreaks]{Justification of Research}

\end{frame}

\begin{frame}[allowframebreaks]{Literature Review}

\end{frame}

\begin{frame}[allowframebreaks]{Characteristics of the Data and Data Sources}

\end{frame}

\begin{frame}[allowframebreaks]{Methodology}

\end{frame}

\begin{frame}[allowframebreaks]{Method Justification}

\end{frame}

\begin{frame}[allowframebreaks]{Results Figures}

\end{frame}

\begin{frame}[allowframebreaks]{Descriptive Statistics}

\end{frame}

\begin{frame}[allowframebreaks]{Results}

\end{frame}

\begin{frame}[allowframebreaks]{Graphs}

\end{frame}

\begin{frame}[allowframebreaks]{Conclusions}

\end{frame}
% ************************
% ***** BIBLIOGRAPHY *****
% ************************
\begin{frame}[allowframebreaks]{Bibliography}
	Here is a reference to a source. For example, \textit{Momentum Contrast for Unsupervised Visual Representation Learning by He et al. (2020)} \cite{he2020momentum} is a placeholder text for a bibliography entry.

	\framebreak

	\scriptsize
	\printbibliography
\end{frame}
% ***************
% ***** END *****
% ***************

\end{document}
