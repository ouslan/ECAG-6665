\documentclass[10pt, oneside]{article}
\usepackage{amsmath, amsthm, amssymb, calrsfs, wasysym, verbatim, bbm, color, graphics, geometry}
\usepackage{authblk}

\geometry{tmargin=.75in, bmargin=.75in, lmargin=.75in, rmargin = .75in}

\newcommand{\R}{\mathbb{R}}
\newcommand{\C}{\mathbb{C}}
\newcommand{\Z}{\mathbb{Z}}
\newcommand{\N}{\mathbb{N}}
\newcommand{\Q}{\mathbb{Q}}
\newcommand{\Cdot}{\boldsymbol{\cdot}}

\newtheorem{thm}{Theorem}
\newtheorem{defn}{Definition}
\newtheorem{conv}{Convention}
\newtheorem{rem}{Remark}
\newtheorem{lem}{Lemma}
\newtheorem{cor}{Corollary}


\title{La asociación entre los desiertos de comida y distintas causas de muerte en Puerto Rico}
\author{Alejandro Ouslan y Julio C. Hernandez}
\affil{USDA y LEADING Hispanics}

\begin{document}

\maketitle
\tableofcontents

\vspace{.25in}

\section{Introducción y Objetivos}

En Puerto Rico, no se ha investigado el impacto de los desiertos de alimentos en la salud de la población. Estos desiertos se definen como áreas donde
hay una notable escasez de acceso a alimentos frescos, saludables y asequibles. Se caracterizan por la falta de supermercados, colmados o mercados de
agricultores que ofrezcan una variedad adecuada de alimentos nutritivos.

\section{Metodologia}

Los datos sobre las distintas causas de muerte relacionadas con condiciones de salud provienen del Departamento de Salud, mientras que la información sobre los
establecimientos se obtiene del QCEW. La base de datos utilizada es un panel que incluye 2,779 observaciones correspondientes al período de 2015Q1 a 2020Q2, desglosadas por códigos ZIP de Puerto Rico.
Además, se incorporaron otros datos comunitarios del American Community Survey. metodología empleada para esta investigación es el estimador Arellano–Bover/Blundell–Bond, que utiliza datos de panel para medir la
correlación de rezagos en la variable independiente.
\[
	\begin{split}
		y_{it} &= \sum_{j=1}^{p} \alpha_j y_{i,t-j} + X_{it} \beta_1 + W_{it} \beta_2 + v_i + \varepsilon \\
		\text{ donde } i &= 1, \ldots, N \text{ y } t = 1, \ldots, T_i
	\end{split}
\]
\section{Resultados}
El estudio reveló una relación negativa entre la cantidad de establecimientos de comida y las enfermedades cancerígenas, circulatorias y respiratorias.
Por otro lado, se encontró una relación positiva entre la inseguridad alimentaria y estas mismas enfermedades. También se observó una correlación entre
los hogares que participan en el programa de Asistencia Nutricional (PAN) y las enfermedades cancerígenas y respiratorias. Además, se identificó que
los hogares de bajos ingresos están correlacionados con un aumento en la incidencia de enfermedades cancerígenas y respiratorias.

\section{Conclusiones}

El estudio encontró una relación negativa entre la cantidad de establecimientos de comida y la incidencia de enfermedades como el cáncer, las enfermedades
circulatorias y respiratorias. Por lo tanto, es fundamental crear incentivos para fomentar la apertura de supermercados, colmados o mercados de agricultores
en las áreas que se consideran desiertos de alimentos. Además, la implementación del programa SNAP podría contribuir a reducir la inseguridad alimentaria en Puerto Rico.

\end{document}
